% Setup -------------------------------

\documentclass[a4paper]{report}
\usepackage[a4paper, total={6in, 10in}]{geometry}
\setcounter{secnumdepth}{3}
\setcounter{tocdepth}{3}

\usepackage{hyperref}
\usepackage{indentfirst}

\usepackage{graphicx}
\usepackage{titlepic}

% Encoding
%--------------------------------------
\usepackage[T1]{fontenc}
\usepackage[utf8]{inputenc}
%--------------------------------------

% Portuguese-specific commands
%--------------------------------------
\usepackage[portuguese]{babel}
%--------------------------------------

% Hyphenation rules
%--------------------------------------
\usepackage{hyphenat}
%--------------------------------------

% Capa do relatório

\title{
	Tolerância a Faltas
	\\ \Large{\textbf{Trabalho Prático}}
	\\ -
	\\ Mestrado em Engenharia Informática
	\\ Universidade do Minho
}

\author{
	\begin{tabular}{ll}
		\textbf{Grupo nº 8}
		\\
		\hline
		PG41080 & João Ribeiro Imperadeiro
        \\
		PG41081 & José Alberto Martins Boticas
		\\
		PG41091 & Nelson José Dias Teixeira
	\end{tabular}
	\vspace{1cm}
	}
	
\date{\today}

\titlepic{
	\vspace{2cm}
	\includegraphics[scale=0.065]{Image/EEUM_logo.png}
}

\begin{document}

\begin{titlepage}
	\maketitle
\end{titlepage}

% Índice

\tableofcontents
\listoffigures

% Introdução

\chapter{Introdução} \label{ch:Introduction}
\large {
	Toda a gente, de uma forma ou de outra, já esteve em contacto com uma loja \textit{online}. Muitos podem mesmo dizer que já dependem deste tipo de serviços para efetuar as suas compras. Por não estarem diretamente relacionadas com uma localização física, estas lojas estão disponíveis para todos, independentemente de onde se encontre fisicamente no mundo.

	Assim, levantam-se alguns problemas relacionados com a implementação destes tipos de serviços, como por exemplo a oferta de um serviço com bom desempenho para todos os clientes, independentemente da sua localização física. Isto leva à necessidade de não depender de apenas um servidor central, distribuindo a disponibilização do serviço por diversos servidores.
	Ora, isto leva a que seja necessário ter cuidados extra nas interações com os clientes, como a manutenção da consistência entre os servidores, para que, no caso em que um destes servidores falhe, o cliente não seja afetado negativamente.

	Posto isto, é-nos proposta a implementação, em \texttt{Java}, de uma versão simplificada de um supermercado \textit{online} distribuído por vários servidores, que por sua vez seja o mais tolerante a faltas possível. Para isso, entre outras ferramentas, será utilizado o protocolo \textit{Spread} para comunicação em grupo, à semelhança do que se sucedeu ao longo das aulas da componente prática desta unidade curricular.

	Relativamente à estrutura do presente documento, é descrito de forma mais detalhada a proposta deste projeto, fazendo-se referência aos requisitos intrínsecos à mesma.
	De seguida, são exibidos todos os aspetos referentes à implementação deste trabalho. Nesta fase, são especificados todos os tópicos alusivos ao cliente, servidor e, ainda, ao \textit{middleware} genérico da aplicação.
	Posteriormente, são evidenciadas todas as valorizações tomadas em consideração no desenvolvimento do mesmo.
	Por fim, são extraídas algumas conclusões da realização deste trabalho, sumariando, globalmente, os objetivos alcançados.
}

\chapter{Descrição do problema e requisitos} \label{ch:ProblemDescriptionRequirements}
\large{
	Tal como foi mencionado no \hyperref[ch:Introduction]{capítulo introdutório} deste documento, o objetivo principal deste projeto prático consiste na implementação em \texttt{Java}, usando o protocolo de comunicação em grupo \textit{Spread}, de um serviço tolerante a faltas.

	Este serviço diz respeito a um supermercado \textit{online} que disponibiliza algumas funcionalidades aos seus clientes. Entra elas destacam-se:
	\begin{itemize}
		\item criar uma encomenda;
		\item iniciar uma compra;
		\item consultar o preço e disponibilidade de um produto;
		\item acrescentar um produto a uma determinada encomenda;
		\item confirmar uma encomenda, indicando se foi concretizada com sucesso.
	\end{itemize}

	O serviço guarda um catálogo contendo uma descrição de cada produto e a quantidade disponível.
	Cada encomenda inclui um ou mais produtos, sendo que só pode ser concretizada com sucesso se todos os produtos estiverem disponíveis.
	Admite-se que existe um tempo limite \texttt{TMAX} para a concretização de uma encomenda. Caso esse tempo seja esgotado e a encomenda ainda não tenha sido efetuada, é cancelada.  Embora seja indesejável, admite-se também que uma encomenda pode ser cancelada unilateralmente pelo sistema.

	Quanto aos requisitos da aplicação são impostos os seguintes:
	\begin{itemize}
		\item par cliente/servidor da interface descrita, replicado para tolerância a faltas;
		\item permitir o armazenamento persistente do estado dos servidores na base de dados \textit{HSQLDB};
		\item transferência de estado para permitir a reposição em funcionamento de servidores sem interrupção do serviço;
		\item implementação de uma interface simplificada para o utilizador, de forma a testar o serviço em causa;
	\end{itemize}

	Tendo em conta todos os pontos referidos acima, procede-se agora à implementação da proposta inerente a este projeto prático.
}

\chapter{Implementação} \label{ch:Implementation}
\large{
	De seguida serão abordadas a forma adotada para implementar várias vertentes deste trabalho. Será apresentado o middleware genérico desenvolvido para servir de base para este trabalho, bem como o uso deste num par cliente/servidor que atinge os objetivos cumpridos.

	\section{\textit{Middleware} genérico} \label{sec:Middleware}
		Se for excluída a lógica de negócio relacionada com o funcionamento do supermercado, temos na essência deste trabalho um conjunto de servidores ligados entre si que gerem uma base de dados em HSQLDB (por recomendação do enunciado). Para além disto, todos os servidores são réplicas perfeitas uns dos outros, ou seja, todos aplicam as mesma alterações às suas bases de dados, o que significa que, na prática, é obtida uma base de dados distribuída por eles. Os papéis dos servidores são simétricos, sendo que qualquer um deles poderá receber pedidos de clientes interessados no serviço disponibilizado.

		Posto isto, foi tomada a decisão de implementar esta arquitetura genérica em classes independentes da lógica de negócio propriamente dita, dando origem a duas classes: ServerConnection e ClientConnection. A primeira é responsável por gerir o conjunto de servidores, sendo iniciada uma instância desta em cada um destes. A segunda é utilizada por clientes para interagir com os servidores de forma transparente.

		Seguidamente, será detalhada cada uma destas classes.

		\subsection{\textit{ServerConnection}} \label{subsec:ServerConnection}
			\subsubsection{Replicação} \label{sssec:Replication}
			\subsubsection{Temporizadores} \label{sssec:Timers}
			\subsubsection{Comunicação entre servidores} \label{sssec:ServerCommunication}
			\subsubsection{Comunicação com clientes} \label{sssec:ClientCommunication}
		\subsection{\textit{ClientConnection}} \label{subsec:ClientConnection}
	
	\section{Servidor - \textit{Supermarket}} \label{sec:Server}
		\subsection{Funcionamento} \label{subsec:ServerWorking}
		\subsection{\textit{CartSkeleton}} \label{subsec:ServerCartSkeleton}
		\subsection{\textit{CatalogSkeleton}} \label{subsec:ServerCatalogSkeleton}
	
	\section{Cliente} \label{sec:Client}
		\subsection{Funcionamento} \label{subsec:ClientWorking}
		\subsection{\textit{CartStub}} \label{subsec:ClientCartStub}
		\subsection{\textit{CatalogStub}} \label{subsec:ClientCatalogStub}
}

\chapter{Valorizações} \label{ch:ProblemDescription}
\large{
	Relativamente ao que é referido no enunciado deste trabalho, são recomendadas algumas valorizações que beneficiam a nota final do mesmo.
	Das valorizações mencionadas, os elementos que compõem este grupo optaram por realizar as seguintes:
	\begin{enumerate}
		\item separação do código relativo ao \textit{middleware} genérico de replicação do código alusivo à aplicação;
		\item garantia do tratamento concorrente de varias operações;
		\item suporte de partições do grupo na ferramenta computacional \textit{Spread};
		\item realização de uma análise de desempenho;
		\item minimização das encomendas canceladas como consequência de faltas ou do funcionamento do mecanismo de replicação;
		\item atualização oportuna do estado dos servidores com recurso ao sistema de base de dados, diminuindo, sempre que possível, o volume da informação copiada.
	\end{enumerate}
}

\chapter{Conclusão} \label{ch:Conclusion}
\large{
	
}

\appendix
\chapter{Observações} \label{ch:Observations}
\begin{itemize}
    \item Documentação \textit{Java} 8:
    \par \textit{\url{https://docs.oracle.com/javase/8/docs/api/}}
	\item \textit{Maven}:
	\par \textit{\url{https://maven.apache.org/}}
	\item \textit{Spread toolkit}:
	\par \textit{\url{http://www.spread.org/index.html}}
	\item \textit{Atomix}:
	\par \textit{\url{https://atomix.io/}}
	\item \textit{HSQLDB}:
	\par \textit{\url{http://hsqldb.org/}}
\end{itemize}

\end{document}