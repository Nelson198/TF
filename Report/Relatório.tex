% Setup -------------------------------

\documentclass[a4paper]{report}
\usepackage[a4paper, total={6in, 10in}]{geometry}
\setcounter{secnumdepth}{3}
\setcounter{tocdepth}{3}

\usepackage{hyperref}
\usepackage{indentfirst}

\usepackage{graphicx}
\usepackage{titlepic}

% Encoding
%--------------------------------------
\usepackage[T1]{fontenc}
\usepackage[utf8]{inputenc}
%--------------------------------------

% Portuguese-specific commands
%--------------------------------------
\usepackage[portuguese]{babel}
%--------------------------------------

% Hyphenation rules
%--------------------------------------
\usepackage{hyphenat}
%--------------------------------------

% Capa do relatório

\title{
	Tolerância a Faltas
	\\ \Large{\textbf{Trabalho Prático}}
	\\ -
	\\ Mestrado em Engenharia Informática
	\\ Universidade do Minho
}

\author{
	\begin{tabular}{ll}
		\textbf{Grupo nº 8}
		\\
		\hline
		PG41080 & João Ribeiro Imperadeiro
        \\
		PG41081 & José Alberto Martins Boticas
		\\
		PG41091 & Nelson José Dias Teixeira
	\end{tabular}
	\vspace{1cm}
	}
	
\date{\today}

\titlepic{
	\vspace{2cm}
	\includegraphics[scale=0.065]{Imagens//EEUM_logo.png}
}

\begin{document}

\begin{titlepage}
	\maketitle
\end{titlepage}

% Índice

\tableofcontents
\listoffigures

% Introdução

\chapter{Introdução} \label{ch:Introduction}
\large {
Toda a gente, de uma forma ou de outra, já esteve em contacto com uma loja online. Muitos podem mesmo dizer que já dependem deste tipo de serviços para efetuar as suas compras. Por não estarem diretamente relacionadas com uma localização física, estas lojas estão disponíveis para todos, independentemente de onde se encontre fisicamente no mundo.

Assim, levantam-se alguns problemas relacionados com a implementação destes tipos de serviços, como por exemplo a oferta de um serviço com bom desempenho para todos os clientes, independentemente da sua localização física. Isto leva à necessidade de não depender de apenas um servidor central, distribuindo a disponibilização do serviço por diversos servidores. Ora, isto leva a que seja necessário ter cuidados extra nas interações com os clientes, como a manutenção da consistência entre os servidores, para que, no caso em que um destes servidores falhe, o cliente não seja afetado negativamente.

Posto isto, é-nos proposto que implementemos, em Java, uma versão simplificada de um supermercado online distribuído por vários servidores, que seja o mais tolerante a faltas possível. Para isso, entre outras ferramentas, será utilizado o protocolo Spread para comunicação em grupo.
}

\chapter{Descrição do problema e requisitos} \label{ch:ProblemDescriptionRequirements}
\large{
	
}

\chapter{Implementação} \label{ch:Implementation}
\large{
	\section{Middleware Genérico} \label{sec:Middleware}
		\subsection{ServerConnection} \label{subsec:ServerConnection}
			\subsubsection{Replicação} \label{subsec:Replication}
			\subsubsection{Temporizadores} \label{subsec:Timers}
			\subsubsection{Comunicação entre servidores} \label{subsec:ServerCommunication}
			\subsubsection{Comunicação com clientes} \label{subsec:ClientCommunication}
		\subsection{ClientConnection} \label{subsec:ClientConnection}
	
	\section{Servidor - Supermarket} \label{sec:Server}
		\subsection{Funcionamento} \label{subsec:ServerWorking}
		\subsection{CartSkeleton} \label{subsec:ServerCartSkeleton}
		\subsection{CatalogSkeleton} \label{subsec:ServerCatalogSkeleton}
	
	\section{Cliente} \label{sec:Client}
		\subsection{Funcionamento} \label{subsec:ClientWorking}
		\subsection{CartStub} \label{subsec:ClientCartStub}
		\subsection{CatalogStub} \label{subsec:ClientCatalogStub}
}

\chapter{Valorizações} \label{ch:ProblemDescription}
\large{
	
}

\chapter{Conclusão} \label{ch:Conclusion}
\large{
	
}

\appendix
\chapter{Observações} \label{ch:Observations}
\begin{itemize}
    \item Documentação \textit{Java} 8:
    \par \textit{\url{https://docs.oracle.com/javase/8/docs/api/}}
	\item \textit{Maven}:
	\par \textit{\url{https://maven.apache.org/}}
	\item \textit{Spread toolkit}:
	\par \textit{\url{http://www.spread.org/index.html}}
	\item \textit{Atomix}:
	\par \textit{\url{https://atomix.io/}}
	\item \textit{HSQLDB}:
	\par \textit{\url{http://hsqldb.org/}}
\end{itemize}


\end{document}